\documentclass[11pt, a4paper]{report}
\usepackage[utf8]{inputenc}
\usepackage[spanish]{babel}
\usepackage{amsmath}
\usepackage{amssymb}
\usepackage{amsfonts}
\usepackage{hyperref}
\usepackage{textcomp}
\hypersetup{
    colorlinks=true,
    linkcolor=black,
    citecolor=green,
    filecolor=magenta,      
    urlcolor=cyan,
}

\title{Estadística 1}
\author{Jorge Miguel Alvarado Reyes}
\date{16 Agosto 2023}

\begin{document}

\maketitle

\tableofcontents

\newpage

\section{Medidas de Tendencia Central}

Las medidas de tendencia central son valores de un conjunto de datos que se encuentran en el centro de los datos ordenados.

\subsection{Media}

Existen dos tipos de media: la aritmética y la ponderada.

La \textbf{media aritmética} se calcula sumando todos los valores y dividiendo por la cantidad de valores:
\[
    \text{Media}(x) = \frac{\sum_{i=1}^{n} x_i}{n}
\]

\textbf{Propiedades}:
\begin{enumerate}
    \item $Media(cx) = c \cdot Media(x)$
    \item $Media(x+c) = Media(x) + c$
\end{enumerate}

\textbf{Ejemplo 4}:
Mostrar que $Media(x+c) = Media(x) + c$.

\textbf{Demostración}:
\[
    \begin{aligned}
        Media(x+c) & = \frac{x_1+c + \ldots + x_n+c}{n}         \\
                   & = \frac{x_1 + \ldots + x_n + n \cdot c}{n} \\
                   & = \frac{x_1 + \ldots + x_n}{n} + c         \\
                   & = Media(x) + c
    \end{aligned}
\]

\textbf{Ejemplo 5}:
Mostrar que $Media(cx) = c \cdot \text{Media}(x)$.

La \textbf{media ponderada} se define como:
\[
    \overline{x} = \frac{\sum_{i=1}^{n} w_i \cdot x_i}{\sum_{i=1}^{n} w_i}
\]

\subsection{Mediana}

La mediana es el valor central que divide a un conjunto de datos ordenados en dos partes iguales. Si $n$ es par, se calcula como:
\[
    Mediana(x) = \frac{x(\frac{n}{2}) + x(\frac{n}{2} + 1)}{2}
\]

% - Estilos -

\subsection{Moda}

Es el valor que mas se repite en un conjunto  de observaciones. \\
\textbf{Ejemplo 6}:

\begin{enumerate}
    \item $[1,2,3,4,5]$ Aqui no existe moda
    \item $[3,4,4,5,5,6]$ Moda = 4.5
    \item $[3,3,4,5,6,6]$ Moda = 3 y 6
    \item $[2,7,7,7,9]$ Moda = 7
\end{enumerate}

\subsection{Media para una serie de frecuencias}

Si $f_1, ... , f_n$ son las frecuencias de la variable $x$. Entonces.
\[
    Mediana(x) = \frac{\sum^{n}_{i=1}f_i \cdot x_i}{\sum^{n}_{i=1}f_i}
\]


\textbf{Ejemplo 7}: \\
Calcula la media para los siguientes valores
\[
    \begin{tabular}{|c|c|}
        \hline
        $x$ & $f_i$ \\
        \hline
        2   & 4     \\
        5   & 1     \\
        6   & 3     \\
        8   & 4     \\
        \hline
    \end{tabular}
\]

\[
    Mediana(x) = \frac{(2\cdot4) + (5\cdot1) + (6\cdot3) + (8\cdot4)}{4+1+3+4} = \frac{8+5+18+32}{12} = \frac{63}{12}
\]

\subsection{Media para datos agrupados}

Sean $f_1, ... , f_n$ las frecuencias de la varible $x$ y $c_1, ... , c_n$ las marcas de clase, entonces: (Marca de clase es un representante)

\[
    Media(x) = \frac{\sum^{n}_{i=1}f_i \cdot c_i}{\sum^{n}_{i=1}f_i}
\]

\textbf{Ejemplo 8}: \\
Calcula la edad promedio para el siguiente conjunto de datos
\[
    \begin{tabular}{|c|c|}
        \hline
        Adulto                    & 25 \\
        \hline
        Adulto de la tercera edad & 10 \\
        \hline
    \end{tabular}
\]

Adulto, edad = [20,65], $c_1 = 43$ \\
Adulto tercera edad, edad = [65,100], $c_1 = 83$\\

25 veces 43 y 10 veces 83

\[
    Mediana(x) = L_{i} t(\frac{\frac{n}{2} - 3 \sum f_i}{f_{mediana}}) \cdot c \\ \\
\]

\noindent $L_i$ = limite inferior de la clase que contiene la mediana\\
$n$ = frecuencia total \\
$\sum f_i$ = suma de las frecuencias menores a la mediana\\
$f_{mediana}$ = Frecuencia de la clase que contiene la mediana\\
$c$ = longitud del intervalo que contiene la mediana \\

\newpage

\section{18 Agosto 2023}

Sitio del curso: https://piazza.com/unam.mx/other/ei20241 \\
codigo de acceso: 150621

\subsection{Breve introduccion a latex}

LaTeX es una herramienta para crear documentos de una gran
calidad tipográfica, en donde los usuarios se ocupan en mayor
medida del contenido del texto en lugar del formato.

\subsubsection{Principales clases de documentos}

\[
    \begin{tabular}{|c|c|}
        \hline
        Clase   & Proposito                                      \\
        \hline
        article & Articulos de revista                           \\
        report  & Textos largos como tesis o reportes            \\
        book    & Libros o documentos con una estructura similar \\
        lette   & cartas                                         \\
        \hline
    \end{tabular}
\]

\subsubsection{Paquetes}

\[
    \begin{tabular}{|c|c|}
        \hline
        Nombre                    & Funcion                                  \\
        \hline
        amsmath, amssymb, amsfont & Permiten el uso de símbolos matemáticos. \\
        babel                     & Escribir en diversos idiomas.            \\
        inputec                   & Codificacion de entradas.                \\
        \hline
    \end{tabular}
\]

\subsubsection{Estructura de un documento}

\begin{verbatim}
    \documentclass[11pt, a4paper]{report}
    \usepackage[utf8]{inputec}
    \usepackage[spanish]{babel}
    \usepackage{amsmath}
    \usepackage{amssymb}
    \usepackage{amsfont}
    
    \title{Titulo}
    \author{Nombre}
    \date{\today}
    \begin{document}
    \maketitle
    ...
    \end{document}
\end{verbatim}

\subsubsection{LaTex en linea}
Crear cuenta en https://es.overleaf.com \\

New project \textrightarrow Blank project \textrightarrow Escribir nombre del documento
\textrightarrow Create

Menu \textrightarrow spell check spanish

\subsubsection{Partes de un documento}

\begin{verbatim}
    \section*{title}
    \subsection*{title}
    \subsubsection*{title}

    \part*{title}
    \chapter*{title}
\end{verbatim}

\subsubsection{Tamaños de fuente}

\begin{verbatim}
    \huge
    \Huge
    \LARGE
    \Large
    \large
    \normalsize
    \small
    \tiny
\end{verbatim}

\subsubsection{Listas numeradas y viñetas}

\begin{verbatim}
    \begin{itemize}[a]
        \item
        \item 
    \end{itemize}

    \begin{enumerate}
        \item
        \item 
    \end{itemize}
\end{verbatim}

\subsubsection{Alineacion de texto}

\begin{verbatim}
    \begin{center}
        ...
    \end{center}
\end{verbatim}

\subsubsection{Composicion de ecuaciones}

\begin{verbatim}
    $x^2+2x+3=0$
\end{verbatim}

\subsubsection{Alinear expresion con algun elemento}

\begin{verbatim}
    \begin{align*}
        c^2 &= a^2 + b^2 \\
        &= 2^2 + 3^2 \\
        &= 13
    \end{align*}
\end{verbatim}

\subsubsection{Tablas}

\begin{verbatim}
    \begin{table}[h]
        \centering
        \begin{tabular}{c | c  c}
            a & b & c\\   
            a & b & c\\   
            a & b & c\\        
        \end{tabular}
    \end{table}

    \begin{table}[h]
        \centering
        \begin{tabular}{| c c c |}
            \hline
            a & b & c\\   
            \hline
            a & b & c\\  
            \hline 
            a & b & c\\
            \hline       
        \end{tabular}
    \end{table}
    
\end{verbatim}

\subsubsection{Como insertar una imagen}

\begin{verbatim}
    \usepackage{graphicx}
    \includegraphics[width = , height = ]{archivo.jpg,png,etc.}
\end{verbatim}

\newpage

\subsection{Clase}

\textbf{Ejemplo 9}:

Encuentran la mediana para las siguientes observaciones

\begin{table}[h]
    \centering
    \begin{tabular}{| c | c | c |}
        \hline
        Intervalo      & Frecuencia & Frecuencia acumulada \\
        \hline
        (118.5,126.5]  & 3          & 3                    \\
        \hline
        (126.5,135.5]  & 5          & 8                    \\
        \hline
        (135.5,144.5]  & 9          & 17                   \\
        \hline
        (144.5,153.5]  & 12         & 29                   \\
        \hline
        (153.5,162.5]  & 5          & 34                   \\
        \hline
        (162.5,171.5]  & 4          & 38                   \\
        \hline
        (171.5, 180.5] & 2          & 40                   \\
        \hline
    \end{tabular}
\end{table}

\noindent $L_i$ = limite inferior de la clase que contiene la mediana\\
$n$ = frecuencia total \\
$\sum f_i$ = suma de las frecuencias menores a la mediana\\
$f_{mediana}$ = Frecuencia de la clase que contiene la mediana\\
$c$ = longitud del intervalo que contiene la mediana \\

\noindent $L_i$ = 144.5\\
$n$ = 40\\
$\sum f_i$ = 17\\
$f_{mediana}$ = 12 \\
$c$ = 153.5 - 144.5 = 9\\

\begin{align*}
    Mediana(x) & = \frac{x_{20} + x_{21}}{2} = 146.75                         \\
               & = L_i + (\frac{\frac{n}{2} - \sum f_i}{f_{mediana}}) \cdot c \\
\end{align*}

\section{21 Agosto 2023}

\subsection{Medidas de posición}



\end{document}